
\chapter{Colouring}
A vertex colouring of a graph $G = (V,E)$ is a map $\phi: V \rightarrow S$ such that $\phi(u) \neq \phi(v) ~~\forall uv \in E(G)$.

$S$, set of colours. Implicitly $\subset \mathbb{N}$.

\bigskip
If $\phi$ is a $k-$coloring, $|S| \leq k$. Also, we define the \textit{chromatic number} $\chi$ as:
\begin{eqnarray}
	\chi (G) := \min \{ k : \exists k-\text{colouring of }G \}
\end{eqnarray}

A graph $G$ with $\chi_{G} = k$ is called $k-$chromatic; if $\chi_{G} \leq q$, we call $G$ $k-$colourable.\\

Note that $k-$colouring is nothing but a vertex parition into $k$ independent sets, now called \textit{colour classes}. The non-trivial 2-colourable graphs, for example, are precisely the bipartite graphs.

	\section{Colouring maps and planar graphs}
		\subsection{The four colour conjecture (1852)}
		\textit{If a graph $G$ is planar, then $\chi_{G} \leq 4$.	\\}
		
		The has been "proved" by Appel and Haken in 1977 using 2000 graph configurations (resulting in 200 pages). but none has really accepted their proof. \\
		
		Then it has been reproved by Robertson, Saners, Seymour, Thomas (1997, 600 configurations, 160 pages), who tried to understand the first proof but couldn't make sense of it. \\
		
		In 2005, Gonthiers (Microsoft research) made a complete formal proof so that it can be verified by computers, which it has.
		
		\subsection{The five colour theorem (Kempe and Heawood, 1890)}
		\textit{If $G$ is planar, $\chi_{G} \leq 5$.\\}
		
		Proof: by induction on $|V(G)|$, valid if $|V(G)| \leq 5$ so let's assume $|V(G)| \geq 6$. 
		\begin{enumerate}
			\item The average degree of $G$ is:
				\begin{eqnarray}
					d(G) = \frac{\sum_{v \in V(G)} d(v)}{|V(G)|} = \frac{2 |E(G)|}{|V(G)|} \leq \frac{2 (3|V(G)| - 6)}{|V(G)|} < 6
				\end{eqnarray}
				Now if this average is strictly less than 6, there is at least one vertex $v$ with $d(v) \leq 5$.
			\item If $d(v) \leq 4$, then consider a 5-colouring of $G-v$, one colour is available at $v$, done.
			\item If $d(v) = 5$, showing that $\chi_{G}$ could be 6 is now quite straightforward, think of removing $v$ and then putting it back in $G$). Now we have a potential problem if we consider a 5-colouring: there must be two distinct non-adjacent vertices $x, y$ in the neighbours of $v$ such that $xy \notin E(G)$. We will first contract $xv$ and $yv$ (thus keeping the planar property, so we can apply induction). By induction, $G / \{ vx, vy \}$ has a 5-colouring $\phi'$. Let:
				\begin{eqnarray}
					\phi(w) &:=& \phi'(w) ~\forall w\ \in V(G) - \{ vx, vy \}\\
					\phi(x) &:=& \phi(v) := \phi(y) = \phi'(\text{contracted vertex})
				\end{eqnarray}
				$\phi$ is not a valid colouring because of the edges $vx, vy$. However, these are the only two problematic edges. Moreover, since $\phi(x) = \phi(y)$, there is some colour $c$ that is not used on any neighbour of $v$, thus setting $\phi(v) := c$ makes $\phi$ a valid 5-colouring.
		\end{enumerate}
		
		
		\subsection{Discharging method (Weinicke, 1904)}
		\textit{One example to prove above's theorem. If $G$ is a plane triangulation and $G$ has minimum degree $5$, then $\exists vw \in E(G)$ such that $d(v) = 5$ and $d(w) \in \{ 5, 6 \}$.\\}
		
		Put a charge of 6-$d(v)$ on each $v \ in (G)$ with:
		\begin{eqnarray}
			\sum_{v \in V(G)} \text{charge}(v) &=& 6 |V(G)| - \sum_{v \ in V(G)} d(v) = 6 |V(G)| - 2 |E(G)|\\
			&=& 6 |V(G)| - 2 (3|V(G)| - 6) \\
			&=& 12 > 0
		\end{eqnarray}
		As we're in a plane triangulation, we can use Euler's formula.\\
		
		One dicharging rule: every degree-5 vertex sends  a charge of $1/5$ to each of its neighbours.\\
		
		Let \text{charge'}$(v)$ denote the new charge of $v$. 
		\begin{eqnarray}
			\sum_{v \in V(G)} \text{charge'}(v) = \sum_{v \in V(G)} \text{charge}(v) = 12 > 0
		\end{eqnarray}
		Let $v \in V(G)$ be such that charge'$(v) > 0$. Now we need to do a simple case analysis:
		\begin{enumerate}
			\item $d(v) \in \{ 5,  6 \}$. Then $v$ received some charge from some neighbour $w$, which thus has degree 5 and hence $vw$ is an edge as desired.
			\item $d(v) \geq 8$. This cannot happen because:
				\begin{eqnarray}
					\text{charge'}(v) &\leq&  \text{charge}(v) + d(v)/5\\
					&=& 6 - d(v) + d(v)/5\\
					&=& 6 - \frac{4}{5}d(v)\\
					&\leq& 6 - \frac{4 \cdot 8}{5} < 0
				\end{eqnarray}
			\item $d(v) = 7$. By the same calculation, $v$ receives charges from at least 6 from its neighbours. With the condition that the new charge must be strictly positive, we can find two neighbours of $v$ that have degree 5 and are adjacent.
		\end{enumerate}
		
	A \textit{clique} is a set of pairwise adjacent vertices. A clique of size 3 is a triangle. We have:
	\begin{eqnarray}
		w(G)  := \text{maximum size of a clique in }G
	\end{eqnarray}
	Clearly, $\chi(G) \geq w(G) ~\forall G$. For example, a cycle of 5 vertices has a clique of size 2 but requires 3 colours. Formally, we have $\chi(C_{2k + 1}) = 3$ but $w(C_{2k + 1}) = 2$, for $k \geq 2$.\\
	
	In fact, there are triangle-free graphs (i.e. $w \leq 2$) with arbitrarily high $\chi$.\\
	
	The \textit{girth} of a graph $G$ is the length of a shortest cycle contained in $G$. Its value is considered infinite if $G$ has no cycle.
	
		\subsection{Theorem (Erdős, 1950)}
		\begin{eqnarray}
			\forall k \geq 1, \exists G : \chi(G) \geq k \wedge \text{girth}(G) \geq k
		\end{eqnarray}
		
		One can always 2-colour a tree. If locally, up until some extent, the graph is a tree but contains a cycle globally, the cycle will enforce the value of $\chi$.\\
		
		The chromatic number is a global phenomenon.
		
	So the clique size is a lower bound, but at times this can be a very bad bound. For bipartite graphs, we have equality, but that's not always so.
	
	\section{Perfect graphs}
		A graph $G$ is \textit{perfect} if $\chi(H) = w(H)$ for all induced subgraphs $H$ of $G$. For example, bipartite graphs are perfect.


\subsection{Weak perfect graph conjecture}
$G$ perfect $\iff \overline{G}$ perfect

\subsection{Strong perfect graph conjecture}
$G$ perfect\\
$\iff$ neither $G$ nor $\overline{G}$ contain odd cycle of length at least $5$
as an induced subgraph

\subsection{Theorem of Lovász, 1992}
$G$ perfect
$\iff \alpha(H)\,\omega(H) \geq |H|
\qquad\forall$ induced subgraph $H$ of $G$

\bigskip
\textit{Stability number}
$\alpha(G) :=$ max size of a stable set (independant set).
